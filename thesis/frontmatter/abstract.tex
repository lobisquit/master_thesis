% Abstract

Internet is currently growing in an unprecedented way, both in terms of users
and offered traffic. %
%
Software-Defined Networks (SDN) will be necessary to ensure a suitable Quality
of Service for users in a more and more complex future, as legacy IP networks
are not up to the task because of their configuration inflexibility. %
%
Such programmable networks have proven successful in many context, from
data-centers to backbone management, but currently no research has analyzed what
happens in small-scale scenarios, such as common city access networks. In this
thesis we try to quantify how much improvement an SDN approach can give to such
a simple infrastructure.

\smallskip

As our case of study, we choose the city of Aachen, located in north-west of
Germany in the state of North Rhine-Westphalia, site of RWTH
university, which this thesis was written in collaboration with. %
%
Since no schematics of its access network were publicly available, we infer its
topology from population density and building maps, setting an optimal location
for each network switch, from the DSLAMs to the backbone router.

% Our approach is inspired by well-known techniques, but significantly improves
% over them both in terms of input instance size and bound with the theoretical
% optimum, provided by an Integer Linear Programming solver.

After this design phase, we devise a strategy to distribute the common bandwidth
among network users in a \emph{fair} way. %
%
Each user is therefore assigned an utility function that links its amount of
resources with perceived Quality of Experience (QoE). %
%
The best synthesis for these individual evaluations is found in literature as
the so-called \emph{Nash arbitration scheme}, equilibrium of the cooperative
allocation game played among users.%
%
This optimal operation point is compared then with \emph{proportional fairness},
the traditional approach for flow control where each user is assigned a
bandwidth proportional to its demands.

Simulating network operations, we show that our approach indeed improves user
QoE with respect to legacy techniques of a significant extent, especially when
demand is high and therefore network management is more challenging for
administrators.

%%% Local Variables:
%%% ispell-local-dictionary: "en"
%%% eval: (flyspell-mode)
%%% TeX-master: "../thesis"
%%% End:
