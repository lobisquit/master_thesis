%!TEX root = ../dissertation.tex
% sommario

Internet sta vivendo in questo periodo una crescita senza precedenti, sia in
termini di utenti che di traffico offerto complessivo.

Le cosiddette Software-Defined Networks (SDN) saranno necessarie in futuro per
garantire le richieste di Quality of Service (QoS) da parte degli utenti, dal
momento che le rete IP tradizionali non sono abbastanza flessibili e
configurabili per la crescente domanda.

Queste reti programmabili hanno trovato applicazione in molti ambiti, dal
\emph{cloud computing} alla gestione di reti backbone, ma al momento nessuno
studio le ha messe alla prova in contesti più comuni, come la rete di accesso di
una città. Questa tesi si pone quindi come obiettivo di studiare i limiti delle
SDN, valutando qual è il loro impatto in contesti più semplici.

\smallskip

Il nostro caso di studio è, nello specifico, la rete di accesso della città di
Aquisgrana, situata nel nord-ovest della Germania, nello stato della
Nord-Renania Westfalia. %
%
Siccome i suoi schematici non sono pubblici, abbiamo ricavato la topologia di
questa rete a partire dalla distribuzione degli edifici e della popolazione, e
posizionando poi in modo ottimale gli switch sulla superficie cittadina.

Ultimata la fase di design, ci siamo posti come obiettivo quello di allocare le
risorse di rete tra i terminali in modo ``fair''.
%
Ad ogni utente è perciò assegnata una funzione utilità, il cui compito è quello
di stabilire la sua Quality of Experience (QoE) a seconda della banda
disponibile.
%
La miglior sintesi di queste valutazioni soggettive è il cosiddetto \emph{Nash
arbitration scheme}, ovvero il punto di equilibrio, secondo la teoria dei
giochi, del problema di allocazione.
%
La strategia proposta è quindi confrontata con quella tradizionale, detta
\emph{proportional fairness}, che assegna ad ogni utente una banda proporzionale
alle sue richieste.

Dopo aver simulato la rete, proveremo che il nostro approccio migliora la
qualità del servizio per gli utenti anche in questo semplice
contesto. L'effetto è rilevante specialmente quando il traffico offerto aumenta,
e quindi la gestione della rete da parte dell'amministratore si complica.

%%% Local Variables:
%%% ispell-local-dictionary: "it"
%%% eval: (flyspell-mode)
%%% End:
